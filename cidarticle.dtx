% \iffalse meta-comment
%<*internal>
\iffalse
%</internal>
%<*internal>
\fi
\def\nameofplainTeX{plain}
\ifx\fmtname\nameofplainTeX\else
  \expandafter\begingroup
\fi
%</internal>
%<*install>
\input docstrip.tex
\keepsilent
\askforoverwritefalse
\declarepreamble\classpre
-----------| -----------------------------------------------------------------
cidarticle:| A class for submissions to the ``Commentarii informaticae didacticae (CID)''
    Author:| Martin Sievers
     Email:| martin.sievers@schoenerpublizieren.de
   License:| Released under the LaTeX Project Public License v1.3c or later
       See:| http://www.latex-project.org/lppl.txt
-----------| -----------------------------------------------------------------
\endpreamble

\def\templatepre{%
\perCent\space !TeX encoding = UTF-8^^J%
\perCent\space !TeX program = pdflatex^^J%
\perCent\space !BIB program = biber^^J%
\perCent\space !TeX spellcheck = en_US^^J%
}

\def\templatepreger{%
\perCent\space !TeX encoding = UTF-8^^J%
\perCent\space !TeX program = pdflatex^^J%
\perCent\space !BIB program = biber^^J%
\perCent\space !TeX spellcheck = de_DE^^J%
}

\postamble

Copyright (C) 2023 by Universitätsverlag Potsdam

This work may be distributed and/or modified under the
conditions of the LaTeX Project Public License (LPPL), either
version 1.3c of this license or (at your option) any later
version.  The latest version of this license is in the file:

http://www.latex-project.org/lppl.txt

This work is "maintained" (as per LPPL maintenance status) by
Martin Sievers.

This work consists of the file  cidarticle.dtx
                                cidarticle.ins
                                README.md
                                CHANGELOG.md
          and the derived files cidarticle.pdf
                                cidarticle.cls
                                cidarticle-author-template.tex
                                cidarticle-example.bib
\endpostamble

\usedir{tex/latex/cidarticle}
\AddGenerationDate
\generate{
  \usepreamble\classpre
  \file{\jobname.cls}{\from{\jobname.dtx}{class}}
}
%</install>
%<install>\endbatchfile
%<*internal>
\usedir{source/latex/cidarticle}
\generate{
  \usepreamble\classpre
  \file{\jobname.ins}{\from{\jobname.dtx}{install}}
}
\nopreamble\nopostamble
\usedir{doc/latex/cidarticle}
\generate{
   \usepreamble\templatepreger
   \file{\jobname-author-template.tex}{\from{\jobname.dtx}{template}}
}
\generate{
  \file{\jobname-example.bib}{\from{\jobname.dtx}{examplebib}}
}
\ifx\fmtname\nameofplainTeX
  \expandafter\endbatchfile
\else
  \expandafter\endgroup
\fi
%</internal>
% \fi
%
% \iffalse
%<*driver>
\ProvidesFile{cidarticle.dtx}
%</driver>
%<class>\NeedsTeXFormat{LaTeX2e}[1999/12/01]
%<class>\ProvidesClass{cid}
%<*class>
    [2023/11/25 v1.1 Official class for submissions to the ``Commentarii informaticae didacticae (CID)'']
%</class>
%<*driver>
\documentclass[a4paper]{ltxdoc}
\usepackage[ngerman,english]{babel}
\usepackage[utf8]{inputenc}
\usepackage[T1]{fontenc}
\usepackage{libertinus}
\usepackage[scaled=0.8]{beramono}
\usepackage[%
   final,%
   tracking=smallcaps,%
   expansion=alltext,%
   protrusion=alltext-nott]{microtype}%
\SetTracking{encoding=*,shape=sc}{50}%
\usepackage{textcomp}
\usepackage{upquote}
\usepackage[final]{listings}
\usepackage{csquotes}
\usepackage[dvipsnames]{xcolor}
%\usepackage{hologo}
\usepackage{dtxdescribe}
\usepackage[%
	pdftitle={cidarticle: Official LaTeX class for submissions to the ``Commentarii informaticae didacticae (CID)''},
   pdfauthor={Martin Sievers},
   urlcolor=blue,%
	linktoc=both,%
	colorlinks=true]{hyperref}
\usepackage[nameinlink,capitalise]{cleveref}

\DeclareFontFamily{U}{MnSymbolC}{}
\DeclareSymbolFont{MnSyC}{U}{MnSymbolC}{m}{n}
\DeclareFontShape{U}{MnSymbolC}{m}{n}{
    <-6>  MnSymbolC5
   <6-7>  MnSymbolC6
   <7-8>  MnSymbolC7
   <8-9>  MnSymbolC8
   <9-10> MnSymbolC9
  <10-12> MnSymbolC10
  <12->   MnSymbolC12%
}{}
\DeclareMathSymbol{\powerset}{\mathord}{MnSyC}{180}

\newcommand{\cid}{\texttt{cidarticle}}
\newcommand{\CID}{\emph{Commentarii informaticae didacticae}}
\lstset{
 basicstyle   = \small\ttfamily,
 gobble       = 2,
 keywordstyle = \color{blue}\bfseries,
 breaklines,
 prebreak = {\mbox{\quad$\hookleftarrow$}},
 language     = [LaTeX]{TeX},
 moretexcs    = {,
   addbibresource,authorrunning,%
   doi,affil,
   ExecuteBibliographyOptions,includegraphics,printbibliography,
 }
 frame        = single,
 backgroundcolor = \color{yellow!60},
 framesep     = 5pt,
 literate={Ö}{{\"O}}1 {Ä}{{\"A}}1 {Ü}{{\"U}}1 {ß}{{\ss}}1 {ü}{{\"u}}1 {ä}{{\"a}}1 {ö}{{\"o}}1
}%
\lstnewenvironment{examplecode}[1][]
{\lstset{#1}}
{}
\providecommand*\env[1]{\texttt{#1}}
\providecommand*\file[1]{\texttt{#1}}
\providecommand*\opt[1]{\texttt{#1}}
\providecommand*\pkg[1]{\textsf{#1}}
\OnlyDescription     %nur Anleitung (ohne Index und History)
\CodelineIndex       %kein Index wenn auskommentiert
\EnableCrossrefs     %kein Index wenn auskommentiert
\RecordChanges       %keine History wenn auskommentiert
\begin{document}
  \DocInput{\jobname.dtx}
\end{document}
%</driver>
% \fi
% \CheckSum{0}
% \CharacterTable
%  {Upper-case    \A\B\C\D\E\F\G\H\I\J\K\L\M\N\O\P\Q\R\S\T\U\V\W\X\Y\Z
%   Lower-case    \a\b\c\d\e\f\g\h\i\j\k\l\m\n\o\p\q\r\s\t\u\v\w\x\y\z
%   Digits        \0\1\2\3\4\5\6\7\8\9
%   Exclamation   \!     Double quote  \"     Hash (number) \#
%   Dollar        \$     Percent       \%     Ampersand     \&
%   Acute accent  \'     Left paren    \(     Right paren   \)
%   Asterisk      \*     Plus          \+     Comma         \,
%   Minus         \-     Point         \.     Solidus       \/
%   Colon         \:     Semicolon     \;     Less than     \<
%   Equals        \=     Greater than  \>     Question mark \?
%   Commercial at \@     Left bracket  \[     Backslash     \\
%   Right bracket \]     Circumflex    \^     Underscore    \_
%   Grave accent  \`     Left brace    \{     Vertical bar  \|
%   Right brace   \}     Tilde         \~}
%
% \changes{v1.0}{2023/10/27}{Official release of first version used for CID 13}
%
% \GetFileInfo{\jobname.dtx}
% \DoNotIndex{\newcommand,\newenvironment}
%
% \title{\textsf{cidarticle} -- Official class for submissions to the\\%
% ``Commentarii informaticae didacticae''\thanks{This file describes version
% \fileversion, last revised \filedate.}}
% \author{Martin Sievers\setcounter{footnote}{6}\thanks{Email:
% martin.sievers@schoenerpublizieren.de}}
% \date{Released \filedate}
%
% \maketitle
%
% \begin{abstract}
% \noindent The \cid{} bundle is used for writing articles to be published in
% the ``Commentarii informaticae didacticae (CID)''. 
% It is based on the class used for the ``Lecture
% Notes in Informatics (LNI)''.
% \end{abstract}
%
% \section{Introduction}
% The \LaTeX\ class for the ``Commentarii informaticae didacticae (CID)''
% is based on the established work for the ``Lecture Notes in Informatics'',
% maintained mainly by Oliver Kopp and Martin Sievers (cf. \url{https://github.com/gi-ev/LNI)}
%
% This is the first public release. I would like to thank 
% the ``Universitätsverlag Potsdam'' (especially Felix Will and Marco Winkler) 
% and the editors of CID 13 (especially Simone Opel)
% for their suggestions and testing.
% 
% All development is done on GitHub (\url{https://github.com/sieversMartin/CID/}).
% You are very welcome to contribute by raising issues or pull requests.
%
% \section{Installation}
% This version of the \cid{} bundle is distributed via
% GitHub and (preferably) CTAN.
%
% The later is the basis for all updates of the two main \TeX{} distributions
% \MiKTeX{} and \TeX{}~Live. Thus the easiest way to get all files needed to
% typeset an article for the \CID{} is to use the package manager of your
% distribution.
%
% There is also a template on Overleaf (\url{https://www.overleaf.com/latex/templates/cid-commentarii-informaticae-didacticae/xgzphtgbbffb}).
%
% For a manual installation please call \texttt{pdflatex cidarticle.dtx} at least
% twice and copy all resulting files (cls, tex, pdf) to your local TEXMF tree.
% Don't forget to update your file name database.
%
% \section{Usage}
% To use the predefined layout for a (German) submission to the \CID{} just
% load the class file as usual with \cs{documentclass\{cid\}}.
%
% The class file loads a bunch of packages which are all part of modern \TeX{}
% distributions. Therefore, if you are confronted
% with a missing package, please try to download and install it using your
% distribution's package manager. Alternatively go to
% \href{http://www.ctan.org}{CTAN} to download missing packages.
%
% The \cid{} class can be used with \pdfLaTeX{} as well as with
% \XeLaTeX{} and \LuaLaTeX{}.
%
% \subsection{Options}\label{sec:options}
% Although the class file includes all layout information for a submission to
% the \CID{}, there are options to adapt the output one way or another.
%
% \DescribeOption{english}A document loading the \CID{} class file uses German
% language adoptions by default. To switch to English, just load the class with
% option \opt{english}. The language influences the hyphenation patterns and 
% terms used in the text.
%
% \DescribeOption{crop}%
% Option \opt{crop} gives you some crop marks (using the package \pkg{crop}) to
% better illustrate the final
% result of your article.
%
% \DescribeOption{nocleveref}When referencing figures, one has to type
% \texttt{Figure\textasciitilde}\cs{ref\marg{label}}. The package \pkg{cleveref}
% reduces the effort by offering the command \cs{cref\marg{label}}. This can be
% used with all floating objects. The package is loaded by default. In case it
% causes issues, one can disable it using with the \opt{nocleveref} option.
%
% \DescribeOption{nohyperref}\pkg{hyperref} is used for colored hyperlinks
% within the articles. If you consider problems or just do not want that
% feature, you can disable it by using the option \opt{nohyperref}.
%
% \DescribeOption{norunningheads}To easily remove all running
% headers from your document, you can use the option \opt{norunningheads}.
%
% \section{Setting up a document}
% You can use the file \texttt{cidarticle-author-template.tex} as a starting point
% for setting up a document for submission. The \cid{} class uses the standard
% ways to build an article.
%
% \subsection{Special meta comments}\label{sec:metadata}
% There is not just one \enquote{\TeX} and one \enquote{bibliography tool}, but
% many different ways to transform a .tex file into a PDF.
% Some \TeX{} editors like \texttt{TeXstudio}, \texttt{TeXmaker} and
% \texttt{TeXshop} support a special set of meta comments to give some 
% information, how to deal with a concrete document.
%
% A typical example looks like:
% \begin{examplecode}
% % !TeX program = pdflatex
% % !BIB program = biber
% % !TeX encoding = UTF-8
% % !TeX spellcheck = en_US
% \documentclass[english]{cidarticle}
% \end{examplecode}
%
% \subsection{Title page}\label{sec:titlepage}
% \DescribeMacro{\title}%
% \DescribeMacro{\subtitle}%
% The title of your work is given using the \cs{title} macro. In addition to
% the title itself, you can add a short title to be used
% in the header of a page:
% \begin{examplecode}[label={lst:title}]
% \title[Short title]{Title}
% \end{examplecode}
%
% You can also add a subtitle by \cs{subtitle\marg{subtitle}}.
%
% \DescribeMacro{\author}\DescribeMacro{\affil}%
% The authors of an article are
% given using an extended \cs{author} macro, which holds not only the name, but also
% email adress and ORCID iD. Moreover the affiliation marker (number) is given as an optional
% argument. Affiliations are added with
% \cs{affil\oarg{number}\marg{information}} where you can use
% \texttt{\textbackslash\textbackslash} to split the address.
% \begin{examplecode}[label={lst:author}]
% \author[1]{Vorname1 Nachname1}{vorname.name@affiliation1.de}{0000-0000-0000-0000}
% \author[2]{Firstname2 Lastname2}{vorname.name@affiliation2.de}{0000-0000-0000-0000}
% \author[1]{Firstname3 Lastname 3}{vorname.name@affiliation1.de}{0000-0000-0000-0000}
% \author[1]{Firstname4 Lastname 4}{vorname.name@affiliation1.de}{0000-0000-0000-0000}% 
% \affil[1]{Universität\\Abteilung\\Straße\\Postleitzahl Ort\\Land}
% \affil[2]{University\\Department\\Address\\Country}
% \end{examplecode}
%
% Leave the third and/or fourth argument empty if there is no email address and/or ORCID iD. 
% Finally \cs{maketitle} will output the formatted title page.
%
% \subsection{Abstract and keywords}
% \DescribeEnv{abstract}\DescribeEnv{keywords}%
% \DescribeMacro{\and}%
% Each article should start with a short (70 to 150 words) abstract and some
% keywords. Please use the environments \env{abstract} and \env{keywords} for
% that purpose:
% \begin{examplecode}
% \begin{abstract}
% Tell the reader what your article is about
% \end{abstract}
% \begin{keywords}
% Give some keywords to categorize your article. You can use \and between two
% keywords to get the correct delimiter (semicolon plus space) automatically.
% \end{keywords}
% \end{examplecode}
%
% \subsection{Page header}\label{sec:pageheader}
% The template automatically sets the page headers according to the
% requirements of \CID. From page~2 onwards, the title and the authors are
% printed. These information have to stay in one line. In case the title is too
% long, use the optional argument for \cs{title}:
% \begin{examplecode}
% \title[short title]{title}
% \end{examplecode}
%
% \subsection{Main text}
% \subsubsection{Headings}
% \DescribeMacro{\section}\DescribeMacro{\subsection}%
% \DescribeMacro{\subsubsection}\DescribeMacro{\paragraph}
% You can use the standard macros \cs{section}, \cs{subsection}, \cs{subsubsection} 
% and \cs{paragraph} sectioning your text.
% \vspace{2\baselineskip}
% \subsubsection{Footnotes}
% \DescribeMacro{\footnote}%
% For adding a footnote, just use \cs{footnote\marg{footnote text}} where
% needed. Please note, that the footnote counter is automatically set to the
% correct value at the beginning of your text, i.\,e. it respects the number
% of affiliations given on the title page.
%
% \subsubsection{Lists}
% \DescribeEnv{itemize}\DescribeEnv{enumerate}%
% The \cid{} class redefines the standard lists environments \env{itemize} and
% \env{enumerate} to meet the requirements of the \CID{}.
%
% Lists can be filled as usual by adding \cs{item} macros.
%
% \subsubsection{Floating objects}
% \DescribeEnv{figure}\DescribeEnv{table}%
% The environments \env{figure} and \env{table} can be used the standard way to
% include graphics or tables resp.
%
% However, please note, that the default placement parameters are changed to
% \opt{htbp} by the class \cid{}. If you need some local adjustment, please use
% the optional argument of both environments (cf.~Listing~\ref{lst:figure}).
%
% \DescribeMacro{\caption}\DescribeMacro{\label}
% A caption should be added by \cs{caption\marg{caption text}}, followed
% immediately by a \cs{label\marg{unique label}} entry. For figures the caption
% should be added below the graphic, for tables place it above \env{tabular}:
% \begin{examplecode}[label={lst:figure}]
% \begin{figure}[tb]
%    \includegraphics{...}
%    \caption{...}
%    \label{...}
% \end{figure}
% \end{examplecode}
%
% If you want to center floats, please \emph{do not} use the \env{center}
% environment, but the macro \cs{centering}, which does not add extra white
% space (cf.~Listing~\ref{lst:table}).
% \begin{examplecode}[label={lst:table}]
% \begin{table}
%    \centering
%    \caption{...}
%    \label{...}
%    \begin{tabular}{lll}
%    ...
%    \end{tabular}
% \end{table}
% \end{examplecode}
%
% \subsubsection{\texorpdfstring{Listings\,/\,Source code}%
%   {Listings/Source code}}
% The \cid{} bundle loads the \pkg{verbatim} and \pkg{listings} package. While
% the former is there for compatability, the later is the standard way of
% integrating source code listings into a \LaTeX{} document.
%
% However, there are currently no config files shipped with the \cid{} bundle.
% Please consult the documentation for help on setting up \pkg{listings} for a
% specific programming language.
%
% \subsubsection{Math}
% For using math the \pkg{amsmath} package is loaded by default. You can load
% package \pkg{mathtools} for additional features.
%
% \subsubsection{Abbreviations and initialisms}
% \DescribeMacro{\eg}\DescribeMacro{\ie}\DescribeMacro{\cf}%
% \DescribeMacro{\etal}%
% To achieve consistent typesetting of common abbreviations, macros are
% predefined by the class. These macros should \emph{consistently} being used
% instead of writing the plain version. For example use \verb|\eg| rather than
% {\verb|e.g.,|}. The macros take care of spacing within and after the
% abbreviations.
% \begin{itemize}
% \item \cs{eg} for e.\,g.
% \item \cs{ie} for i.\,e.
% \item \cs{cf} for cf.
% \item \cs{etal} for et~al.
% \end{itemize}
%
% \DescribeMacro{\cidinitialism} You can add your own initialisms by stating
% \cs{cidinitialism\marg{\textbackslash initialism\_macro}\marg{text}} in the
% preamble. Then use the new macros within your text wherever needed.
%
% \subsection{Bibliography}\label{sec:bibliography}
% The \cid{} class uses \pkg{biblatex} by default together with the 
% \pkg{biblatex} style \enquote{lni} provided by 
% \href{https://github.com/gi-ev/biblatex-lni}{\pkg{biblatex-lni}}.
% However, you have to add information on the bib
% file(s) in your document using \cs{addbibresource\marg{Bib file(s)}} and call
% \cs{printbibliography} at the end of your document.
%
% Please note, that the \cid{} class sets
% \texttt{biber} as the default bibliography tool. \texttt{biber} is part of
% both major \TeX{} distributions and can easily be used within most \TeX{}
% editors, e.\,g. by using special meta data as described in
% \cref{sec:metadata}.
%
% \begin{examplecode}
% % !TeX program = pdflatex
% % !BIB program = biber
% \documentclass[]{cidarticle}
% ...
% \addbibresource{FILENAME.bib}
% ...
% \begin{document}
% ...
% \printbibliography
% ...
% \end{document}
% \end{examplecode}
%
% \section{Trouble shooting}
% If you have any problems using the class file please head to
% \href{https://github.com/egeerardyn/awesome-LaTeX/blob/master/README.md}%
% {the awesome \LaTeX{} list}.
%
% \section{Bugs and feature request}
% If you find a bug or have a feature request, please contact me.
% You can open an ``issue'' at the \href{https://github.com/sieversMartin/CID/}{GitHub website}.
%
% \StopEventually{^^A
%  \PrintChanges
%  \PrintIndex
% }
%
% \section{Implementation}
%
%    \begin{macrocode}
%<*class>
%    \end{macrocode}
%    \begin{macrocode}
\def\@clearglobaloption#1{%
  \def\@tempa{#1}%
  \def\@tempb{\@gobble}%
  \@for\next:=\@classoptionslist\do
    {\ifx\next\@tempa
       \message{Cleared  option \next\space from global list}%
     \else
       \edef\@tempb{\@tempb,\next}%
     \fi}%
  \let\@classoptionslist\@tempb
  \expandafter\ifx\@tempb\@gobble
    \let\@classoptionslist\@empty
  \fi}
%
\DeclareOption{latin1}{\PassOptionsToPackage{latin1}{inputenc}}
\DeclareOption{utf8}{\PassOptionsToPackage{utf8}{inputenc}}
\DeclareOption{ansinew}{\PassOptionsToPackage{ansinew}{inputenc}}
\newif\ifcidenglish
\cidenglishfalse
\DeclareOption{english}{\cidenglishtrue\@clearglobaloption{english}}
\newif\ifusehyperref
\usehyperreftrue
\DeclareOption{nohyperref}{\usehyperreffalse}
\newif\ifusecleveref
\useclevereftrue
\DeclareOption{nocleveref}{\useclevereffalse}
\newif\ifusebiblatex
\usebiblatextrue
\DeclareOption{biblatex}{\usebiblatextrue}
\DeclareOption{nobiblatex}{\usebiblatexfalse}
\newif\ifcrop
\cropfalse
\DeclareOption{crop}{\croptrue}
\newif\ifnorunningheads
\DeclareOption{norunningheads}{\norunningheadstrue}
\ExecuteOptions{utf8}
\DeclareOption*{\PassOptionsToClass{\CurrentOption}{scrartcl}}
\ProcessOptions\relax
%
\RequirePackage{iftex}
\PassOptionsToPackage{dvipsnames}{xcolor}
\PassOptionsToPackage{fleqn}{amsmath}
\LoadClass[10pt,a4paper,twoside]{article}
\ifPDFTeX
   \RequirePackage{cmap}
   \RequirePackage{inputenc}
   \RequirePackage[T1]{fontenc}
   \RequirePackage[full]{textcomp}
\else
   \RequirePackage{fontspec}
\fi%
%
\ifcidenglish
   \RequirePackage[ngerman,UKenglish,USenglish,english]{babel}
\else
   \RequirePackage[english,UKenglish,USenglish,ngerman]{babel}
   \babelprovide[hyphenrules=ngerman-x-latest]{ngerman}
   \ClassInfo{cidarticle}{Using latest German hyphenation patterns}%
\fi%
% Hint by http://tex.stackexchange.com/a/321067/9075 -> enable "= as dashes
\useshorthands*{"}
\addto\extrasenglish{\languageshorthands{ngerman}}
\addto\extrasUKenglish{\languageshorthands{ngerman}}
\addto\extrasUSenglish{\languageshorthands{ngerman}}
%    \end{macrocode}
% Load old Times variant (explictly wanted by publisher)
%    \begin{macrocode}
\ifPDFTeX
   \RequirePackage{mathptmx}
   \RequirePackage[%
      final,%
      tracking=smallcaps,%
      expansion=alltext,%
      protrusion=alltext-nott]{microtype}%
\else
   \setmainfont{NimbusRomNo9L}[%
      Extension         = .otf,%
      UprightFont       = *-Reg,%
      BoldFont          = *-Med,%
      ItalicFont        = *-RegIta,%
      BoldItalicFont    = *-MedIta,%
      Ligatures         = {Common},%
      ]%
   \RequirePackage[%
      final,%
      protrusion=alltext-nott]{microtype}%
%    \end{macrocode}
% When using \hologo{LuaLaTeX} we can activate \pkg{selnolig}
%    \begin{macrocode}
   \ifluatex
      \ifcidenglish
         \RequirePackage[english]{selnolig}%
      \else
         \RequirePackage[ngerman]{selnolig}%
      \fi%
   \fi%
\fi%
\SetTracking{encoding=*,shape=sc}{50}%
%    \end{macrocode}
% Introduce \cs{powerset} - hint by \url{http://matheplanet.com/matheplanet/nuke/html/viewtopic.php?topic=136492&post_id=997377}
%    \begin{macrocode}
\DeclareFontFamily{U}{MnSymbolC}{}
\DeclareSymbolFont{MnSyC}{U}{MnSymbolC}{m}{n}
\DeclareFontShape{U}{MnSymbolC}{m}{n}{
    <-6>  MnSymbolC5
   <6-7>  MnSymbolC6
   <7-8>  MnSymbolC7
   <8-9>  MnSymbolC8
   <9-10> MnSymbolC9
  <10-12> MnSymbolC10
  <12->   MnSymbolC12%
}{}
\DeclareMathSymbol{\powerset}{\mathord}{MnSyC}{180}
%    \end{macrocode}
%    \begin{macrocode}
% Support for CC icons
%\RequirePackage{ccicons}
% Support for \cs{ifdefempty}
\RequirePackage{etoolbox}
%    \end{macrocode}
%    \begin{macrocode}
\RequirePackage[oldcommands]{ragged2e}
%    \end{macrocode}
%    \begin{macrocode}
\newlength{\doihoffset}
\newlength{\doivoffset}
\ifcrop
   \RequirePackage[
     paperheight=220truemm,paperwidth=155truemm,
     text={110truemm,176truemm},
     inner=25truemm,
     outer=20truemm,
     top=24.6truemm,
     bottom=19truemm,
     nomarginpar,
     headsep=7truemm,
     headheight=13.5bp, 
     driver=none]
     {geometry}
   \RequirePackage[a4,center,frame,info]{crop}
   \renewcommand*\CROP@@info{{%
      \global\advance\CROP@index\@ne
      \def\x{\discretionary{}{}{\hbox{\kern.5em---\kern.5em}}}%
      \advance\paperwidth-20\p@
      \dimen@4pt
      \ifx\CROP@pagecolor\@empty
      \else
      \advance\dimen@\CROP@overlap
      \fi
      \hb@xt@\z@{%
         \hss
         \vbox to\z@{%
            \centering
            \hsize\paperwidth
            \vss
            \normalfont
            \normalsize
            \expandafter\csname\CROP@font\endcsname{%
               ``\jobname''\x
               \the\year/\the\month/\the\day\x
               \CROP@time\x
               page\kern.5em\thepage\x
               \#\the\CROP@index
               \strut
            }%
            \vskip\dimen@
         }%
         \hss
      }%
   }}%
   \setlength{\doihoffset}{1.45cm}
   \setlength{\doivoffset}{1.2cm}
\else
   \RequirePackage[%
      paperheight=220truemm,paperwidth=155truemm,%
      text={110truemm,176truemm},%
      inner=25truemm,%
      outer=20truemm,%
      top=24.6truemm,%
      bottom=19truemm,%
      nomarginpar,%
      headsep=7truemm,%
      headheight=13.5bp,% 
      driver=none]%
   {geometry}
   %\setlength{\doihoffset}{4.2cm}
   %\setlength{\doivoffset}{4.3cm}
\fi%
%    \end{macrocode}
% We change font sizes to get the correct baseline skips
%    \begin{macrocode}
\renewcommand\normalsize{%
   \@setfontsize\normalsize{10bp}{13.5bp}%
   \abovedisplayskip 10\p@ \@plus2\p@ \@minus5\p@
   \abovedisplayshortskip \z@ \@plus3\p@
   \belowdisplayshortskip 6\p@ \@plus3\p@ \@minus3\p@
   \belowdisplayskip \abovedisplayskip
   \let\@listi\@listI}
\normalsize
\ifx\MakeRobust\@undefined \else
\MakeRobust\normalsize
\fi
\DeclareRobustCommand\small{%
   \@setfontsize\small{9bp}{13.5bp}%
   \abovedisplayskip 8.5\p@ \@plus3\p@ \@minus4\p@
   \abovedisplayshortskip \z@ \@plus2\p@
   \belowdisplayshortskip 4\p@ \@plus2\p@ \@minus2\p@
   \def\@listi{\leftmargin\leftmargini
      \topsep 4\p@ \@plus2\p@ \@minus2\p@
      \parsep 2\p@ \@plus\p@ \@minus\p@
      \itemsep \parsep}%
   \belowdisplayskip \abovedisplayskip
}
\DeclareRobustCommand\footnotesize{%
   \@setfontsize\footnotesize{8bp}{9.6bp}%
   \abovedisplayskip 6\p@ \@plus2\p@ \@minus4\p@
   \abovedisplayshortskip \z@ \@plus\p@
   \belowdisplayshortskip 3\p@ \@plus\p@ \@minus2\p@
   \def\@listi{\leftmargin\leftmargini
      \topsep 3\p@ \@plus\p@ \@minus\p@
      \parsep 2\p@ \@plus\p@ \@minus\p@
      \itemsep \parsep}%
   \belowdisplayskip \abovedisplayskip
}
\DeclareRobustCommand\scriptsize{\@setfontsize\scriptsize\@viipt\@viiipt}
\DeclareRobustCommand\tiny{\@setfontsize\tiny\@vpt\@vipt}
\DeclareRobustCommand\large{\@setfontsize\large{12bp}{14.4bp}}
\DeclareRobustCommand\Large{\@setfontsize\Large{13bp}{15.6bp}}
\DeclareRobustCommand\LARGE{\@setfontsize\LARGE{14bp}{16.8bp}}
\DeclareRobustCommand\huge{\@setfontsize\LARGE{16bp}{19.2bp}}
\DeclareRobustCommand\Huge{\@setfontsize\LARGE{20bp}{24bp}}
%    \end{macrocode}
%    \begin{macrocode}
\RequirePackage[autostyle]{csquotes}
\ifusebiblatex
   \RequirePackage[%
      backend=biber,% UTF-8 support
      style=LNI,    % The GI style - see https://www.ctan.org/pkg/biblatex-lni
      natbib=true,   % Required for \Citet
      autolang=other%
   ]{biblatex}
   \renewcommand{\bibfont}{\normalfont\small}
   \defbibheading{bibliography}[\iflanguage{english}{Bibliography}{Literaturverzeichnis}]{%
      \section*{#1}\markboth{#1}{#1}\pdfbookmark[1]{\iflanguage{english}{Bibliography}{Literaturverzeichnis}}{bibliography\thecidarticle}}
   \patchcmd{\biburlsetup}%
      {\def\UrlBigBreaks{\do\:\do\-}}%
      {\def\UrlBigBreaks{\do\:\do\-\do\/}}%
      {}%
      {}%
%    \end{macrocode}
% \begin{macro}{\citeauthor}
%    \begin{macrocode}
  % Enable hyperlinked authors when using \citeauthor
  % Source: http://tex.stackexchange.com/a/75916/9075
  \DeclareCiteCommand{\citeauthor}%
    {\boolfalse{citetracker}%
     \boolfalse{pagetracker}%
     \usebibmacro{prenote}}%
    {\ifciteindex%
       {\indexnames{labelname}}%
       {}%
     \printtext[bibhyperref]{\printnames{labelname}}}%
    {\multicitedelim}%
    {\usebibmacro{postnote}}%
% \end{macro}
%    \begin{macrocode}
\fi%
%    \end{macrocode}
%    \begin{macrocode}
\RequirePackage{amsmath}
\RequirePackage{graphicx}
\RequirePackage{eso-pic}
\RequirePackage{grffile}
\RequirePackage{fancyhdr}
\RequirePackage{listings}
\RequirePackage{booktabs}
\RequirePackage{tabularx}
%    \end{macrocode}
% We fix the basewidth for lstlistings:
% The default setting of listings with ``fixed columns'' has a space 0.6em
% wide, while the characters in TX Typewriter (as in Computer Modern
% Typewriter) are 0.5em wide.
% Source: https://tex.stackexchange.com/a/179072/9075
%    \begin{macrocode}
\lstset{%
   basicstyle=\ttfamily,%
   columns=fixed,%
   basewidth=.5em,%
   xleftmargin=0.5cm,%
   captionpos=b,%
   upquote}%
%    \end{macrocode}
% Ragged bottom -- verhindert die Ausdehnung der Seite = Veränderung der
% Abstände
%    \begin{macrocode}
\def\thisbottomragged{\def\@textbottom{\vskip\z@ plus.0001fil
\global\let\@textbottom\relax}}
%    \end{macrocode}
% Seitenzahlen -- Größe der Box
%    \begin{macrocode}
\renewcommand\@pnumwidth{2.5em}
\renewcommand\@tocrmarg{2.55em}
\renewcommand\@dotsep{2.5}
\def\@dottedtocline#1#2#3#4#5{%
  \ifnum #1>\c@tocdepth \else
    \vskip \z@ \@plus.2\p@
    {\leftskip #2\relax \rightskip \@tocrmarg \advance\rightskip by 0pt plus 2cm
               \parfillskip -\rightskip \pretolerance=10000
     \parindent #2\relax\@afterindenttrue
     \interlinepenalty\@M
     \leavevmode
     \@tempdima #3\relax
     \advance\leftskip \@tempdima \null\nobreak\hskip -\leftskip
     {#4}\nobreak
     \leaders\hbox{$\m@th
        \mkern \@dotsep mu\hbox{.}\mkern \@dotsep
        mu$}\hfill
     \nobreak
     \hb@xt@\@pnumwidth{\hfil\normalfont \normalcolor #5}%
     \par}%
  \fi}
%    \end{macrocode}
% \begin{macro}{\title}
%    \begin{macrocode}
%\newcommand{\@shorttitle}{}%
\newcommand{\@toctitle}{}%
\newcommand*{\@titlerunning}{}%
\renewcommand{\title}{\@dblarg\@@title}
\def\@@title[#1]#2{%
   \gdef\@title{#2}%
   \renewcommand{\@toctitle}{#2}%
   \renewcommand*{\@titlerunning}{#1}%
}
%    \end{macrocode}
% \end{macro}
% \begin{macro}{\subtitle}
%    \begin{macrocode}
\newcommand{\subtitle}[1]{\gdef\@subtitle{#1}}
%    \end{macrocode}
% \end{macro}
% \begin{macro}{\author}
%    \begin{macrocode}
\gdef\and{\texorpdfstring{\unskip,\ }{, }}
\renewcommand{\author}{\@dblarg\@@author}
\newcommand{\@tocauthor}{}%
\newcommand{\@authorrunning}{}%
\def\@@author[#1]#2{%
   \DeclareRobustCommand{\@author}{%
      #2%
   }%
   \renewcommand{\@tocauthor}{#1}%
   %\protected@edef\@tocauthor{%
   %   #1%
   %}%
   \renewcommand{\@authorrunning}{#1}%
}
%    \end{macrocode}
% \end{macro}
% \begin{macro}{\authorrunning}
%    \begin{macrocode}
\newcommand*{\authorrunning}[1]{%
   \renewcommand*{\@authorrunning}{#1}}
%    \end{macrocode}
% \begin{macro}{\titlerunning}
%    \begin{macrocode}
\newcommand*{\titlerunning}[1]{%
   \renewcommand*{\@titlerunning}{#1}}%
%    \end{macrocode}
% \end{macro}
% \begin{macro}{\email}
%    \begin{macrocode}
\newcommand*{\email}[1]{\ifusehyperref\href{mailto:#1}{\protect\nolinkurl{#1}}\else\protect\nolinkurl{#1}\fi}
%    \end{macrocode}
% \end{macro}
% \begin{macro}{\orcid}
%    \begin{macrocode}
\newcommand*{\orcid}[1]{%
   \ifusehyperref\unskip~\orcidlink{#1}\,\href{https://orcid.org/#1}{https://orcid.org/#1}%
   \else\orcidlink{}\,https://orcid.org/#1\fi}
%    \end{macrocode}
% \end{macro}
% \begin{macro}{\doi}
%    \begin{macrocode}
\newcommand{\@doi}{}
\newcommand{\doi}{%
      \begingroup\catcode`\_12 \doi@i}
\newcommand{\doi@i}[1]{%
      \gdef\@doi{#1}\endgroup}
%    \end{macrocode}
% \end{macro}
%    \begin{macrocode}
\newcounter{cidarticle}
\@addtoreset{section}{cidarticle}%
\@addtoreset{footnote}{cidarticle}%
\@addtoreset{figure}{cidarticle}%
\@addtoreset{table}{cidarticle}%
%    \end{macrocode}
% Title: Kopie aus article.cls mit anderem \thispagestyle
%    \begin{macrocode}
\renewcommand\maketitle{\par%
   \begingroup
    \renewcommand\thefootnote{\@arabic\c@footnote}%
%    \def\@makefnmark% keine Einrückung der Fußnoten eingestellt
%		 {\@textsuperscript{\normalfont\@thefnmark}}%
%		 \long\def\@makefntext##1{%
%		 \@setpar{\@@par
%		 \@tempdima = \hsize
%		 \advance\@tempdima -1em
%		 \parshape \@ne 0.15cm \@tempdima}%
%		 \par\parindent 5mm \noindent
%		 \hb@xt@\z@{\hss{\normalfont\@thefnmark}}\hspace{5mm}##1}
    \if@twocolumn
      \ifnum \col@number=\@ne
        \@maketitle
      \else
        \twocolumn[\@maketitle]%
      \fi%
    \else
      \newpage
      \global\@topnum\z@% Prevents figures from going at top of page.
      \@maketitle
    \fi%
    \ifnorunningheads
      \thispagestyle{empty}
    \else
      \thispagestyle{plain}
    \fi%
    \@thanks
  \endgroup
  \ifusehyperref
      \HyXeTeX@CheckUnicode
      \HyPsd@PrerenderUnicode{\@tocauthor}%
      \pdfstringdef\@pdfauthor{\@tocauthor}%
      \HyXeTeX@CheckUnicode
      \HyPsd@PrerenderUnicode{\@title}%
      \pdfstringdef\@pdftitle{\@title}%
  \fi%
  %\global\let\thanks\relax
  %\global\let\maketitle\relax
  %\global\let\@maketitle\relax
  \global\let\@thanks\@empty
  \global\let\@author\@empty
  %\global\let\@date\@empty
  \global\let\@title\@empty
  \global\let\@subtitle\@empty
  %\global\let\title\relax
  %\global\let\author\relax
  %\global\let\date\relax
  %\global\let\and\relax
}
%    \end{macrocode}
%    \begin{macrocode}
\def\@maketitle{%
  \clearpage\begingroup\pagestyle{empty}\cleardoublepage\endgroup%
  \markboth{\@authorrunning}{\@titlerunning}%
  \addtocontents{toc}{\protect\setcounter{tocdepth}{2}}%
  \phantomsection%
  \addcontentsline{toc}{chapter}{\protect\numberline{}{\@toctitle}}%
  \addtocontents{toc}{\protect\contentsline{subsection}{\@tocauthor}{}{}}%
  \addtocontents{toc}{\protect\setcounter{tocdepth}{0}}%
  \refstepcounter{cidarticle}%
  \null%
  \vskip-16.8bp%\spacebeforesection%
  \vskip 11mm%
  \raggedright% Linksbündig
  \let\footnote\thanks
    {\LARGE\bfseries\@title\par}%
    \ifx\@subtitle\empty\else
      \ifx\@subtitle\undefined\else
         \vskip \baselineskip%
         {\normalsize\bfseries\@subtitle\par}%
      \fi%
    \fi%
    \vskip\baselineskip% Abstand nach dem Titel
    {\normalsize%
      \lineskip .5em%
        \@author
      \par}%
    \vskip 2\baselineskip% Abstand vor dem Abstract
}%
%    \end{macrocode}
% \begin{environment}{abstract}
%    \begin{macrocode}
\newcommand*{\@addmargin}[2][\@tempa]{%
   \list{}{%
      \if@tempswa
         \def\@tempa{\leftmargin}%
         \setlength{\leftmargin}{#2}%
         \setlength{\rightmargin}{#1}%
      \else
         \def\@tempa{\rightmargin}%
         \setlength{\rightmargin}{#2}%
         \setlength{\leftmargin}{#1}%
      \fi
      \setlength{\listparindent}{\parindent}%
      \setlength{\itemsep}{\parskip}%
      \setlength{\itemindent}{\z@}%
      \setlength{\@tempskipa}{\topsep}%
      \setlength{\topsep}{\z@}%
      \setlength{\parsep}{\parskip}%
      \setlength{\@tempskipb}{\partopsep}%
      \setlength{\partopsep}{\z@}%
      \let\makelabel\@gobble
      \setlength{\labelwidth}{\z@}%
      \advance\@listdepth\m@ne
   }%
   \item\nobreak\ignorespaces
}

\renewenvironment*{abstract}{%
   \renewcommand{\abstractname}{Abstract}%
   \@addmargin{\parindent}\small\noindent\ignorespaces{\bfseries\abstractname:\ }%
}{\advance\@listdepth\@ne
  \endlist
  \par\vspace{\baselineskip}}
%    \end{macrocode}
% \end{environment}
% \begin{environment}{keywords}
%    \begin{macrocode}
\newif\ifkeywords
\newenvironment{keywords}%
   {\global\keywordstrue\normalsize%
    \def\and{\unskip;\space}%
    \noindent\ignorespaces{\bfseries Keywords:\ }}%
   {\par\global\keywordsfalse}
\let\@RIGsection\section
\pretocmd\@startsection{%
   \ifkeywords\ClassError{cidarticle}%
      {keywords is an environment, not a macro}%
      {Please change \string\keywords\space to an environment}%
      \keywordsfalse%
   \fi%
}{}{}%
%    \end{macrocode}
% \end{environment}
% Section headings
%    \begin{macrocode}
\renewcommand{\section}{\@startsection{section}{1}{\z@}%
  {-2\baselineskip}{\baselineskip}{\normalfont\large\bfseries}}
\renewcommand{\subsection}{\@startsection{subsection}{2}{\z@}%
  {-2\baselineskip}{\baselineskip}{\normalfont\normalsize\bfseries}}
\renewcommand{\subsubsection}{\@startsection{subsubsection}{3}{\z@}%
   {-2\baselineskip}{\baselineskip}{\normalfont\normalsize\bfseries}}
\renewcommand{\paragraph}{\@startsection{paragraph}{4}{\z@}%
   {1\baselineskip}{-1em}{\normalfont\normalsize\bfseries}}
\newlength{\spacebeforesection}
\setlength{\spacebeforesection}{\dimexpr\topsep+11mm\relax}
\newcommand{\CIDsection}[1]{%
   \section*{\rule{0pt}{\spacebeforesection}#1}%
   \markboth{#1}{#1}%
   \pdfbookmark[1]{\iflanguage{english}{Bibliography}{Literaturverzeichnis}}{bibliography\thecidarticle}%
}
\newcommand{\CIDVorwort}[1]{\section*{\rule{0pt}{\spacebeforesection}#1}\markboth{#1}{#1}}
%    \end{macrocode}
% Bildunterschriften
%    \begin{macrocode}
\RequirePackage{caption}
\captionsetup[figure]{style=base,font=footnotesize,position=below}
\captionsetup[table]{style=base,font=small,position=above}
\captionsetup[lstlisting]{style=base,font=footnotesize,position=below}
\newcommand{\ifwithinfigure}[2]{%
   \ifx\@currenvir\empty
      \PackageWarningNoLine{ifwithinfigure}{Environment name not available}%
   \else
      \ifx\@currenvir\figure
         #1%
      \else
         #2%
      \fi
   \fi
}
\setlength{\intextsep}{\baselineskip}%Abstand nach der Grafik
\setlength{\abovecaptionskip}{.5\baselineskip}
\setlength{\belowcaptionskip}{%
   \ifwithinfigure{2\baselineskip}{0pt}%
}
%    \end{macrocode}
%    \begin{macrocode}
\renewcommand\tableofcontents{%
   \clearpage\thispagestyle{plain}
   \pdfbookmark[0]{\contentsname}{toc}%
   \CIDVorwort{\contentsname}%
   \@starttoc{toc}%
}

\def\@part[#1]#2{%
   \clearpage\begingroup\pagestyle{empty}\cleardoublepage\endgroup
   \thispagestyle{empty}
   \pdfbookmark[-1]{#2}{part\thepage}%
   \addtocontents{toc}{\protect\contentsline{part}{#1}{}{}}
   \null\vfil
   {\centering
    \interlinepenalty \@M
    \normalfont
    \fontsize{20bp}{20bp}\selectfont\bfseries #2%
    \markboth{}{}\par}%
   \vfil\null\newpage
   \null
   \thispagestyle{empty}%
   \newpage}

\renewcommand*\l@part[2]{%
   \ifnum \c@tocdepth >-2\relax
     \addpenalty\@secpenalty
     \vspace*{2\baselineskip}%
     \setlength\@tempdima{3em}%
     \begingroup
       \parindent \z@ \rightskip \@pnumwidth
       \parfillskip -\@pnumwidth
       {\leavevmode
           \fontsize{13bp}{13.5bp}\selectfont #1\hfil
           \hb@xt@\@pnumwidth{\hss #2%
           \kern-\p@\kern\p@}}\par
        \nobreak
        \if@compatibility
          \global\@nobreaktrue
          \everypar{\global\@nobreakfalse\everypar{}}%
       \fi
     \endgroup
     \vspace*{2\baselineskip}%
   \fi}
\renewcommand*\l@section[2]{%
   \ifnum \c@tocdepth >\z@
      \addpenalty\@secpenalty
      \setlength\@tempdima{0pt}%
      \begingroup
         \parindent \z@ \rightskip \@pnumwidth
         \parfillskip -\@pnumwidth
         \leavevmode
         \advance\leftskip\@tempdima
         \hskip -\leftskip
         #1\nobreak\mdseries
         \leaders\hbox{$\m@th
            \mkern \@dotsep mu\hbox{.}\mkern \@dotsep
            mu$}\hfill
         \nobreak\hb@xt@\@pnumwidth{\hss #2}\par
      \endgroup
   \fi}
\let\l@chapter\l@section%   
\renewcommand*\l@subsection[2]{%
   \ifnum \c@tocdepth >\z@
      \addpenalty\@secpenalty
      \setlength\@tempdima{0pt}%
      \begingroup
         \parindent \z@ \rightskip \@pnumwidth
         \parfillskip -\@pnumwidth
         \leavevmode
         \advance\leftskip\@tempdima
         \hskip -\leftskip
         \itshape #1\nobreak\hfil
         \nobreak\hb@xt@\@pnumwidth{\hss #2%
         \kern-\p@\kern\p@}\par
      \endgroup
   \fi}

%    \end{macrocode}
% Take care of floats
%    \begin{macrocode}
\def\fps@figure{htbp}
\def\fnum@figure{\figurename~\thefigure}
\def\@floatboxreset{%
        \reset@font
        \small
        \@setnobreak
        \@setminipage
}%
\setcounter{topnumber}{10}% maximale Anzahl gleitender Objekte am Seitenanfang
\setcounter{bottomnumber}{10}% maximale Anzahl gleitender Objekte am Seitenende
\renewcommand{\topfraction}{1.0}% Anteil den gleitende Objekte am Seitenanfang einnehmen dürfen
\renewcommand{\bottomfraction}{1.0}% Anteil den gleitende Objekte am Seitenende einnehmen dürfen
%    \end{macrocode}
% Tables
%    \begin{macrocode}
\def\fps@table{htbp}
\def\fnum@table{\tablename~\thetable}
\renewcommand{\arraystretch}{1.1}
%    \end{macrocode}
% Indention for equations with fleqn option
%    \begin{macrocode}
\setlength{\mathindent}{0.5cm}
%    \end{macrocode}
% Indention for verbatim listings
%    \begin{macrocode}
\RequirePackage{verbatim}
\def\verbatim@processline{\hskip0.5cm\the\verbatim@line\par}
%    \end{macrocode}
% Set rule width und correct size
%    \begin{macrocode}
\renewcommand\footnoterule{%
  \kern-3\p@
  \hrule\@width 2.6cm
  \kern5.6\p@}
\RequirePackage[hang,stable,bottom,belowfloats]{footmisc}
\setlength{\footnotemargin}{5mm} % Abstand Fußnotenzähler/Fußnotentext
\def\@cidmakefnmark{\rlap{\normalfont\@thefnmark}}%
\long\def\@makefntext#1{%
   \ifFN@hangfoot
   \bgroup
   \setbox\@tempboxa\hbox{%
      \ifdim\footnotemargin>0pt
         \hb@xt@\footnotemargin{\@cidmakefnmark\hss}%
      \else
         \@cidmakefnmark
      \fi%
   }%
   \leftmargin\wd\@tempboxa
   \rightmargin\z@
   \linewidth \columnwidth
   \advance \linewidth -\leftmargin
   \parshape \@ne \leftmargin \linewidth
   \@totalleftmargin \leftmargin
   \footnotesize
   \@setpar{{\@@par}}%
   \leavevmode
   \llap{\box\@tempboxa}%
   \parskip\hangfootparskip\relax
   \parindent\hangfootparindent\relax
   \footnotelayout#1%
   \ifFN@hangfoot
      \par\egroup
   \fi%
\fi%
}
%    \end{macrocode}
%    \begin{macrocode}
\setlength{\parindent}{5mm}
%    \end{macrocode}
% Set symbols for itemize
%    \begin{macrocode}
\renewcommand{\labelitemi}{\textbullet}
\renewcommand*\itemize{%
  \ifnum \@itemdepth >\thr@@\@toodeep\else
  	\setlength{\labelsep}{0.70cm}%
    \advance\@itemdepth\@ne
    \edef\@itemitem{labelitem\romannumeral\the\@itemdepth}%
    \expandafter
    \list
      \csname\@itemitem\endcsname
      {\def\makelabel##1{\hss\llap{##1}}%
       %\setlength{\itemsep}{8pt}%
       \setlength{\parsep}{-2pt}}%
  \fi}
%    \end{macrocode}
% and numbered lists
%    \begin{macrocode}
  \renewcommand{\labelenumii}{\alph{enumii})}
  \renewcommand*\enumerate{%
  \ifnum \@enumdepth >\thr@@
      \@toodeep
    \else
		\setlength{\labelsep}{0.70cm}%Abstand zur Aufzählungsnummer
      \advance\@enumdepth \@ne
      \edef\@enumctr{enum\romannumeral\the\@enumdepth}%
    \fi
    \@ifnextchar[{\@enumlabel@{\@enumerate@}[}{\@enumerate@}}
  \def\@enumerate@{%
    \expandafter\list\csname label\@enumctr\endcsname{%
      \usecounter{\@enumctr}%
      \def\makelabel##1{\hss\llap{##1}}
		\setlength{\labelsep}{0.6cm} %Einrückung des Aufzählungszeichens
      %\setlength{\itemsep}{8pt}%
      \setlength{\parsep}{-2pt}}
  }%
\usepackage{enumitem}
\setlist{noitemsep}
%    \end{macrocode}
% \begin{macro}{\andname}
%    \begin{macrocode}
\newcommand{\andname}{}
%    \end{macrocode}
% \end{macro}
%    \begin{macrocode}
\addto\captionsngerman{%
   \def\andname{und}%
   \def\figurename{Abb.}%
   \def\tablename{Tab.}%
   \def\lstlistingname{List.}%
   \def\bibname{Literaturverzeichnis}%
   \def\refname{Literaturverzeichnis}%
}%
%    \end{macrocode}
%    \begin{macrocode}
\addto\captionsenglish{%
  \def\andname{and}%
  \def\figurename{Fig.}%
  \def\tablename{Tab.}%
  \def\lstlistingname{List.}%
  \def\bibname{Bibliography}%
  \def\refname{Bibliography}%
}%
\addto\captionsUKenglish{%
   \def\andname{and}%
   \def\figurename{Fig.}%
   \def\tablename{Tab.}%
   \def\lstlistingname{List.}%
   \def\bibname{Bibliography}%
   \def\refname{Bibliography}%
}%
\addto\captionsUSenglish{%
   \def\andname{and}%
   \def\figurename{Fig.}%
   \def\tablename{Tab.}%
   \def\lstlistingname{List.}%
   \def\bibname{Bibliography}%
   \def\refname{Bibliography}%
}%
%    \end{macrocode}
% \begin{macro}{\startpage}
%    \begin{macrocode}
\newcommand*{\startpage}[1]{\setcounter{page}{#1}}
%    \end{macrocode}
% \end{macro}
% \begin{macro}{\booktitle}
%    \begin{macrocode}
\def\@bookshorttitle{}
\newcommand{\booktitle}{\@dblarg\@@booktitle}
\def\@@booktitle[#1]#2{\gdef\@bookshorttitle{#1}\gdef\@booktitle{#2}}
%    \end{macrocode}
% \end{macro}
% \begin{macro}{\booksubtitle}
\newcommand{\booksubtitle}[1]{\gdef\@booksubtitle{#1}}
% \end{macro}
% \begin{macro}{\editor}
%    \begin{macrocode}
\newcommand*{\@editor}{}
\newcommand*{\editor}[1]{\renewcommand{\@editor}{#1}}
%    \end{macrocode}
% \end{macro}
%    \begin{macrocode}
\newcommand*{\@yearofpublication}{\the\year}
\newcommand*{\yearofpublication}[1]{\renewcommand*{\@yearofpublication}{#1}}
%    \end{macrocode}
% set-up for header and footer
%    \begin{macrocode}
\renewcommand{\sectionmark}[1]{}%
\renewcommand{\subsectionmark}[1]{}%
\fancypagestyle{plain}{%
      \fancyhead{}% Löscht alle Kopfzeileneinstellungen
      \renewcommand{\headrulewidth}{0.0pt}% Linie unter Kopfzeile
   }%
\ifnorunningheads
   \pagestyle{empty}%
\else%
   \pagestyle{fancy}%
   \fancyhead[RE]{\footnotesize\leftmark{}}%
   \fancyhead[LO]{\footnotesize\rightmark{}}%
   \fancyhead[LE,RO]{\thepage}%
   \fancyfoot{}% Löscht alle Fußzeileneinstellungen
   \renewcommand{\headrulewidth}{0.4pt}%Linie unter Kopfzeile
   \fancyheadoffset[ol]{0.8mm}%
   \fancyheadoffset[el]{0.8mm}%
   \fancyheadoffset[or]{-0.8mm}%  
   \fancyheadoffset[er]{-0.8mm}%
\fi%
%    \end{macrocode}
%    \begin{macrocode}
\RequirePackage{url}
\urlstyle{same}
%    \end{macrocode}
% improve wrapping of URLs - hint by http://tex.stackexchange.com/a/10419/9075
%    \begin{macrocode}
\g@addto@macro{\UrlBreaks}{\UrlOrds}
%    \end{macrocode}
%    \begin{macrocode}
   \def\UrlBigBreaks{\do\:\do\-}%
   \def\UrlBigBreaks{\do\:\do\-\do\/}
%    \end{macrocode}
%    \begin{macrocode}
\RequirePackage{xspace}
%    \end{macrocode}
%    \begin{macrocode}
\RequirePackage{etoolbox}
\RequirePackage{authblk}
\newcommand{\@authlisthead}{}
\newtoks\@temptokenb
\newtoks\@temptokenc
\renewcommand\Authsep{, }
\renewcommand\Authands{\iflanguage{ngerman}{ und }{, and }}
\renewcommand\Authand{\iflanguage{ngerman}{ und }{ and }}

\renewcommand\author[4][]%
{\ifnewaffil\addtocounter{affil}{1}%
   \edef\AB@thenote{\arabic{affil}}\fi
   \if\relax#1\relax\def\AB@note{\AB@thenote}\else\def\AB@note{#1}%
   \ifcsundef{@emailsandorcids\AB@note}{\csgdef{@emailsandorcids\AB@note}{}}{}%
   \setcounter{Maxaffil}{0}\fi
   \ifnum\value{authors}=0\def\@firstauthor{#2}\fi
   \ifnum\value{authors}>1\relax
   \@namedef{@sep\number\c@authors}{\Authsep}\fi
   \addtocounter{authors}{1}%
   \begingroup
   \let\protect\@unexpandable@protect \let\and\AB@pand
   \def\thanks{\protect\thanks}\def\footnote{\protect\footnote}%
   \@temptokena=\expandafter{\AB@authors}%
   \@temptokenb=\expandafter{\AB@authors}%
   {\def\\{\protect\\[\@affilsep]\protect\Affilfont
         \protect\AB@resetsep}%
      \xdef\AB@author{\AB@blk@and#2}%
      \ifnewaffil\gdef\AB@las{}\gdef\AB@lasx{\protect\Authand}\gdef\AB@as{}%
      \xdef\AB@authors{\the\@temptokena\AB@blk@and}%
      \else
      \xdef\AB@authors{\the\@temptokena\AB@as\AB@au@str}%
      \global\let\AB@las\AB@lasx\gdef\AB@lasx{\protect\Authands}%
      \gdef\AB@as{\Authsep}%
      \fi
      \gdef\AB@au@str{#2}}%
   \@temptokena=\expandafter{\AB@authlist}%
   \@temptokenb=\expandafter{\@authlisthead}%
   \let\\=\authorcr
   \xdef\AB@authlist{\the\@temptokena
      \protect\@nameuse{@sep\number\c@authors}%
      \protect\Authfont#2\AB@authnote{\AB@note}}%
   \xdef\@authlisthead{\the\@temptokenb
      \protect\@nameuse{@sep\number\c@authors}%
      \protect\Authfont#2}% 
   \endgroup
   \ifnum\value{authors}>2\relax
   \@namedef{@sep\number\c@authors}{\Authands}\fi
   \ifcsempty{@emailsandorcids\AB@note}%
   {\csgappto{@emailsandorcids\AB@note}{%
         \if\relax#3\relax\else\email{#3}\fi\if\relax#4\relax\else\ \orcid{#4}\fi}%
   }%
   {\csgappto{@emailsandorcids\AB@note}{%
         \if\relax#3\relax\else\ |\ \email{#3}\fi\if\relax#4\relax\else\ \orcid{#4}\fi}}%
   
   \newaffilfalse
}

\renewcommand\@author{\ifx\AB@affillist\AB@empty\AB@author\else
   \ifnum\value{affil}>\value{Maxaffil}\def\rlap##1{##1}%
   \AB@authlist\AB@affillist
   \else\AB@authors\fi\fi}

\renewcommand\affil[2][]%
{\newaffiltrue\let\AB@blk@and\AB@pand
   \if\relax#1\relax\def\AB@note{\AB@thenote}\else\def\AB@note{#1}%
   \setcounter{Maxaffil}{0}\fi
   \begingroup
   \let\protect\@unexpandable@protect
   \def\thanks{\protect\thanks}\def\footnotetext{\protect\footnotetext}%
   \@temptokena=\expandafter{\AB@authors}%
   {\def\\{\protect\\\protect\Affilfont}\xdef\AB@temp{#2}}%
   \xdef\AB@authors{\the\@temptokena\AB@las\AB@au@str
      \protect\\[\affilsep]\protect\Affilfont\AB@temp}%
   \gdef\AB@las{}\gdef\AB@au@str{}%
   {\def\\{, \ignorespaces}\xdef\AB@temp{#2}}%
   \@temptokena=\expandafter{\AB@affillist}%
   \xdef\AB@affillist{\the\@temptokena 
      \footnotetext[\AB@note]{%
         \raggedright\AB@temp\ifcsempty{@emailsandorcids\AB@note}{}{, \csuse{@emailsandorcids\AB@note}}}%
   }
   \endgroup
}
\def\maketitle{%
   \ifnum\value{authors}>2 
   \authorrunning{\@firstauthor\ et\ al.}%
   \else
   \authorrunning{\@authlisthead}
   \fi%
   \AB@maketitle%
}
%    \end{macrocode}
%    \begin{macrocode}
\ifusehyperref
   \AddToHook{env/document/begin}[cidarticle/loadhyperref]{%
      \RequirePackage{hyperref}
      \RequirePackage[startatroot]{bookmark}
      \RequirePackage{colorprofiles}
      \ifluatex
        \usepackage{luatex85}
      \fi
      \RequirePackage[a-2b,mathxmp]{pdfx}[2018/12/22]
      \hypersetup{%
         colorlinks=false,%
         %allcolors=black,%
         linkcolor=BrickRed,
         citecolor=Green,
         filecolor=Mulberry,
         urlcolor=NavyBlue,
         menucolor=BrickRed,
         runcolor=Mulberry,
         linkbordercolor=BrickRed,
         citebordercolor=Green,
         filebordercolor=Mulberry,
         urlbordercolor=NavyBlue,
         menubordercolor=BrickRed,
         runbordercolor=Mulberry,
         bookmarks=true,
         bookmarksnumbered = true,
         bookmarksopen=true,
         bookmarksdepth=5,
         %pdfsubject = {Technical Report},
         %pdfkeywords = {},
         %pdfcreator = {HPI},
         %pdfmetalang = {en-US},
         %pdfproducer = {CID},
         %pdflang = {en-US},
         pdfdisplaydoctitle = true,
         pdfpagemode = UseOutlines,
         pdfpagelayout = SinglePage,
         pdfview = Fit,
         linktoc=all%
      }      
      \renewcommand{\theHsection}{\thecidarticle-\arabic{section}}%
      \RequirePackage{orcidlink}%
%    \end{macrocode}
% enables correct jumping to figures when referencing
%    \begin{macrocode}
      \RequirePackage[all]{hypcap}%
   }%
   \DeclareHookRule{env/document/begin}{cidarticle/loadhyperref}{before}{biblatex}
\else
   \RequirePackage{tikz}
   \usetikzlibrary{svg.path}
   \definecolor{orcidlogocol}{HTML}{A6CE39}
   \tikzset{
      orcidlogo/.pic={
         \fill[orcidlogocol] svg{M256,128c0,70.7-57.3,128-128,128C57.3,256,0,198.7,0,128C0,57.3,57.3,0,128,0C198.7,0,256,57.3,256,128z};
         \fill[white] svg{M86.3,186.2H70.9V79.1h15.4v48.4V186.2z}
         svg{M108.9,79.1h41.6c39.6,0,57,28.3,57,53.6c0,27.5-21.5,53.6-56.8,53.6h-41.8V79.1z M124.3,172.4h24.5c34.9,0,42.9-26.5,42.9-39.7c0-21.5-13.7-39.7-43.7-39.7h-23.7V172.4z}
         svg{M88.7,56.8c0,5.5-4.5,10.1-10.1,10.1c-5.6,0-10.1-4.6-10.1-10.1c0-5.6,4.5-10.1,10.1-10.1C84.2,46.7,88.7,51.3,88.7,56.8z};
      }
   }
   
   %% Reciprocal of the height of the svg whose source is above.  The
   %% original generates a 256pt high graphic; this macro holds 1/256.
   \newcommand{\@OrigHeightRecip}{0.00390625}
   
   %% We will compute the current X height to make the logo the right height
   \newlength{\@curXheight}
   
   \DeclareRobustCommand\orcidlink[1]{%
      \setlength{\@curXheight}{\fontcharht\font`X}%
      \mbox{%
         \begin{tikzpicture}[yscale=-\@OrigHeightRecip*\@curXheight,
         xscale=\@OrigHeightRecip*\@curXheight,transform shape]
         \pic{orcidlogo};
         \end{tikzpicture}%
      }{}}
   
   \providecommand{\texorpdfstring}[2]{#2}%
   \providecommand{\pdfbookmark}[3][]{\relax}%
   \providecommand{\phantomsection}{\relax}
\fi%
%    \end{macrocode}
%    \begin{macrocode}
\ifusecleveref%
   \AtEndPreamble{%
      \ifcidenglish
        \RequirePackage[capitalise,nameinlink]{cleveref}
        \crefname{section}{Sect.}{Sect.}
        \Crefname{section}{Sect.}{Sect.}
      \else
        \RequirePackage[ngerman,nameinlink]{cleveref}
      \fi%
      \crefname{figure}{\figurename}{\figurename}
      \Crefname{figure}{\figurename}{\figurename}
      \crefname{listing}{\lstlistingname}{\lstlistingname}
      \Crefname{listing}{\lstlistingname}{\lstlistingname}
      \crefname{table}{\tablename}{\tablename}
      \Crefname{table}{\tablename}{\tablename}
   }%
\fi%
%    \end{macrocode}
%    \begin{macrocode}
%\def\and{\texorpdfstring{\unskip\hspace{-0.42em},\hspace{.6em}}{, }}%
%\def\and{\texorpdfstring{\,,\ }{, }}%
%    \end{macrocode}
%    \begin{macrocode}
\newcommand*{\cid@abbrv}[1]{#1\@\xspace}
\newcommand*{\cidabbrv}[2]{\gdef#1{\cid@abbrv{#2}}}
\newcommand*{\cid@initialism}[1]{\textsc{#1}\xspace}
\newcommand*{\cidinitialism}[2]{\gdef#1{\cid@initialism{#2}}}
\newcommand*{\ie}{\cid@abbrv{i.\,e.}}
\newcommand*{\eg}{\cid@abbrv{e.\,g.}}
\newcommand*{\cf}{\cid@abbrv{cf.}}
\newcommand*{\etal}{\cid@abbrv{et~al.}}
\newcommand*{\OMG}{\cid@initialism{omg}}
\newcommand*{\BPM}{\cid@initialism{bpm}}
\newcommand*{\BPMN}{\cid@initialism{bpmn}}
\newcommand*{\BPEL}{\cid@initialism{bpel}}
\newcommand*{\UML}{\cid@initialism{uml}}
%    \end{macrocode}
% bibliography
%    \begin{macrocode}
%\renewenvironment{thebibliography}[1]
%{\iflnienglish\selectlanguage{english}\else\selectlanguage{ngerman}\fi
%   \section*{\refname}%
%   \bgroup\small%
%   \list{\@biblabel{\@arabic\c@enumiv}}%
%   {\settowidth\labelwidth{\@biblabel{#1}}%
%      \leftmargin\labelwidth
%      \advance\leftmargin\labelsep
%      \@openbib@code
%      \usecounter{enumiv}%
%      \let\p@enumiv\@empty
%      \renewcommand\theenumiv{\@arabic\c@enumiv}}%
%   \sloppy
%   \clubpenalty4000
%   \@clubpenalty \clubpenalty
%   \widowpenalty4000%
%   \sfcode`\.\@m}
%{\def\@noitemerr
%   {\@latex@warning{Empty `thebibliography' environment}}%
%   \endlist\egroup}
%    \end{macrocode}
%    \begin{macrocode}
\frenchspacing
\tolerance 1414
\hbadness 1414
\emergencystretch 1.5em
\hfuzz 0.3pt
\widowpenalty=10000
\displaywidowpenalty=10000
\clubpenalty=9999
\interfootnotelinepenalty=9999
\brokenpenalty=2000
\vfuzz \hfuzz
\raggedbottom
%    \end{macrocode}
%    \begin{macrocode}
%</class>
%    \end{macrocode}
%\Finale
%
%\iffalse
%
%<*template>
%%% Um einen Artikel auf deutsch zu schreiben, genügt es die Klasse ohne
%%% Parameter zu laden.
\documentclass[]{cidarticle}
%%% To write an article in English, please use the option ``english'' in order
%%% to get the correct hyphenation patterns and terms.
%%% \documentclass[english]{cidarticle}
%%%
%%% Die CID-Klasse nutzt biblatex, so dass man die .bib-Datei angeben muss:
\addbibresource{cidarticle-example.bib} %Beispiel-Bibliography

\usepackage{blindtext} %nur zu Testzwecken
%%
\begin{document}
%%% Mehrere Autoren werden durch \and voneinander getrennt.
%%% Die Fußnote enthält die Adresse sowie eine E-Mail-Adresse.
%%% Das optionale Argument (sofern angegeben) wird jeweils für das 
%%% Inhaltsverzeichnis *und* den Kolumnentitel im Kopf der Seite verwendet.
%%% Die Angaben für den Kolumnentitel können mittels optionalen Arguments
%%% für \title bzw. \author gesetzt werden.
\title[Ein Kurztitel wenn der Titel z.B. Fußnoten enthält]{Ein sehr langer Titel über mehrere Zeilen mit sehr vielen
Worten und noch mehr Buchstaben}
%%%\subtitle{Untertitel / Subtitle} % if needed
\author[1]{Vorname1 Nachname1}{vorname.name@affiliation1.de}{0000-0000-0000-0000}
\author[2]{Firstname2 Lastname2}{vorname.name@affiliation2.de}{0000-0000-0000-0000}
\author[1]{Firstname3 Lastname 3}{vorname.name@affiliation1.de}{0000-0000-0000-0000}
\author[1]{Firstname4 Lastname 4}{vorname.name@affiliation1.de}{0000-0000-0000-0000}
\affil[1]{Universität\\Abteilung\\Straße\\Postleitzahl Ort\\Land}
\affil[2]{University\\Department\\Address\\Country}

\maketitle

\begin{abstract}
Die Zusammenfassung sollte etwa 70 bis 150 Worte umfassen und besteht in der Regel aus einem Absatz.
\end{abstract}
\begin{keywords}
Schlagwort1 \and Schlagwort2 %Keyword1 \and Keyword2
\end{keywords}
%%% Beginn des Artikeltexts
\section{Überschrift der Ebene 1}
\blindtext[2]

\subsection{Referenzen auf Einträge in der Bibliografie}
Jetzt ein paar Referenzen auf \cite{Bernhard2017} oder auch \cite{ANKOM2014} sowie auf eine Liste \cites[17]{Anderson2012}{Berges2013}{Anderson2001}.

\subsection{Grafiken}
Hier kommt eine Grafik:
\begin{figure}
   \includegraphics[width=\textwidth]{example-image-a}
   \caption{Eine Beschriftung}
   \label{fig:Beispielgrafik}
\end{figure}

Auf die Grafik kann man dann verweisen: \cref{fig:Beispielgrafik}.

\subsection{Mehr Text}
\blindtext[18]

\subsection{Überschrift der Ebene 2}
\blindtext[5]

%%% Ausgabe der Bibliographie (biber vorher aufrufen)
\printbibliography
\end{document}
%</template>
%<*examplebib>
@collection{Anderson2001,
   address = {New York and others},
   title = {A taxonomy for learning, teaching, and assessing},
   subtitle = {A revision of Bloom's taxonomy of educational objectives (gekürzte Version)},
   isbn = {978-0-321-08405-7},
   publisher = {Longman},
   editor = {Anderson, Lorin W. and Krathwohl, David R.},
   year = {2001},
}

@article{Anderson2012,
   title    = {Design-based research},
   subtitle = {A decade of progress in education research?},
   author   = {Anderson, Terry and Shattuck, Julie},
   journal  = {Educational researcher},
   volume   = {41},
   number   = {1},
   pages    = {16-25},
   year     = {2012},
   publisher= {Sage Publications Sage CA},
   location = {Los Angeles, CA}
}

@inproceedings{Berges2013,
   author   = {Berges, Marc and Hubwieser, Peter and Magenheim, Johannes and Bender, Elena and Bröker, Kathrin and Margaritis-Kopecki, Melanie and Neugebauer, Jonas and Schaper, Niclas and Schubert, Sigrid and Ohrndorf, Laura},
   title = {Developing a Competency Model for Teaching Computer Science in Schools},
   year     = {2013},
   isbn     = {9781450320788},
   publisher= {Association for Computing Machinery},
   location = {New York, NY, USA},
   doi      = {10.1145/2462476.2465607},
   booktitle= {Proceedings of the 18th ACM Conference on Innovation and Technology in Computer Science Education},
   pages    = {327},
   location = {Canterbury, England, UK},
   eventtitle = {ITiCSE '13}
}

@book{Bernhard2017,
   title    = {Durch Europäisierung zu mehr Durchlässigkeit?: Veränderungsdynamiken des Verhältnisses von beruflicher Bildung zur Hochschulbildung in Deutschland und Frankreich},
   author   = {Bernhard, Nadine},
   year     = {2017},
   publisher= {Verlag Barbara Budrich}
}

@online{ANKOM2014,
   title    = {ANKOM -- Übergänge von der beruflichen in die hochschulische Bildung},
   url      = {http://ankom.dzhw.eu/},
   urldate  = {2021-09-20},
   author   = {{Deutsches Zentrum für Hochschul- und Wissenschaftsforschung (GmbH)}},
   year     = {2014}
}
%</examplebib>
%    \end{macrocode}
%\fi
